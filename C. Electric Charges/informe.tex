\documentclass{article}
\usepackage[utf8]{inputenc}
\usepackage{graphicx}
\usepackage{url}
\usepackage{hyperref}
\usepackage[a4paper]{geometry}
\geometry{top=1.0cm, bottom=2.0cm, left=1.25cm, right=1.25cm}
\usepackage{amsmath}

\title{DAA-C.Electric Charges\\
 Problem 623/C}
\author{Abel Molina S\'anchez C411}
\date{Julio 2022}

\begin{document}

\maketitle

\section{Descripci\'on del problema}

\textbf{Descripci\'on completa:} \url{https://codeforces.com/problemset/problem/623/C}\\

\textit{Problema} Se tienen $n$ puntos con coordenadas enteras $(x_i,y_i)$. Cada punto se proyectar\'a en uno de los ejes $X(x_i,0)$ o $Y(y_i,0)$. Se quiere conocer cual es el cuadrado del menor di\'ametro posible entre los puntos despu\'es de esta operaci\'on.\\

\textbf{Entrada:}\\

L\'inea 1: $n$ - $(1\leq n\leq 100000)$: n\'umero de puntos\\

Siguientes $n$ l\'ineas: en cada uno dos enteros $x_i$ $y_i$ las coordenadas del i-\'esimo punto. No hay dos puntos coincidentes.\\

\textbf{Salida:}\\

L\'inea 1: un entero, el cuadrado del m\'inimo di\'ametro posible entre los puntos.

\section{An\'alisis del problema}


\textit{Notaci\'on:} Denotaremos $D_2(p_1,p_2)$ al cuadrado del di\'ametro entre dos puntos $p_1$ y $p_2$. \\


\textit{Cota de la soluci\'on:} Tenemos que cada punto ser\'a proyectado en alguno de los ejes, con los cual, hay dos distribuciones de puntos que siempres son posibles: 1- colocarlos todos en el eje $X(x_i,0)$, o colocarlos todos en el eje $Y(0,y_i)$. Con lo cual si tomamos:\\ $x\_min= min$ $x_i \forall i$\\
$x\_max= max$ $x_i \forall i$\\
$y\_min= min$ $y_i \forall i$\\
$y\_max= max$ $y_i \forall i$\\
tenemos que los siguientes $D_2$ ser\'an siempre posibles de alcanzar:\\
$D_2y = (y\_max-y\_min)^2$\\
$D_2x = (x\_max-x\_min)^2$\\

Con lo cual si hacemos $MIND_2 = min(D_2y,D_2x)$, tenemos que cualquier respuesta \'optima estar\'a en el intervalo $[0,MIND_2]$.\\

\textit{Distancia entre los ejes:} A la hora de calcular los di\'ametros habr\'a dos casos, el di\'amtro entre elementos del mismo eje ($X-X$, $Y-Y$), o entre ejes distintos($X-Y)$. En el caso de un mismo eje, como se anula una coordenada, de la f\'ormula de distancia nos queda: $D_2 = (p_1.c-p_2.c)^2$. En el caso de cruce de ejes, tendremos $D_2 = (p_1.c1)^2 + (p_2.c2)^2$, donde en este \'ultimo es v\'alido se\~nalar que es la suma de las distancias respectivas al $(0,0)$(M\'as adelante se menciona este elemento)\\

\textit{Relaci\'on de los di\'ametros:} En una distribuci\'on cualquiera de los puntos luego de proyectarlos en sus respectivos ejes, vamos a tener los siguientes di\'ametros que definen el di\'ametro del conjunto:\\
Con los puntos definidos en el apartado anterior:\\
$D1 = D_2(max_x, min_x)= d(max_x, min_x)^2$\\
$D2 = D_2(max_y, min_y)= d(max_y, min_y)^2$\\
$D3 = D_2(max_x, min_y)= d(max_x, min_y)^2$\\
$D4 = D_2(max_x, max_y)= d(max_x, max_y)^2$\\
$D5 = D_2(min_x, min_y)= d(min_x, min_y)^2$\\
$D6 = D_2(min_x, max_y)= d(min_x, max_y)^2$\\

Tenemos que el di\'ametro del conjunto va a ser el $max(di)$, ya que: supongamos que existe un punto que no es de los mencionados que forma parte de un di\'ametro m\'as grande. Tenemos que ese punto tiene que estar proyectado en alguno de los ejes, sin p\'erdida de generalidad digamos que lo est\'a en el eje $X$. Luego, es claro que este punto no puede tener una distancia mayor con ning\'un punto del eje $X$ que la que tienen $x\_min, x\_max$, precisamente porque de tenerla ser\'ia una contradicci\'on con el hecho de que $x\_min$ y $x\_max$ son extremos de este conjunto. Entonces, queda el caso de que forme un di\'ametro con un punto del eje $Y$ tal que este di\'ametro sea mayor que los mencionados. Pero ning\'un punto del eje $Y$ va a estar a m\'as distancia del eje $X$ que el m\'aximo entre $y\_min, y\_max$, porque nuevamente ser\'ia una contradicci\'on con la extremalidad de dichos puntos en el eje $Y$. Luego, dicho di\'amtero tendr\'ia que ser entre el punto y el m\'aximo(absoluto) de $y\_max$ y $y\_min$, pero este di\'ametro es menor que los di\'ametros que forman el mayor extremo de $X$ con dicha $y$, porque de lo contrario, $|punto|>|max_x|$ lo cual es una contradicci\'on. Con lo cual, el mayor di\'ametro del conjunto est\'a dado por los di\'ametros mencionados anteriormente.\\


\section{Soluciones}

\textit{Fuerza bruta:} La primera soluci\'on de fuerza bruta pasa por explorar el universo de soluciones al completo. Para esto es necesario generar todas las posibles distribuciones de los puntos en los ejes. Por cada punto se tienen dos opciones, eje $X$ o eje $Y$, como tenemos $n$ puntos, por el principio de la multiplicaci\'on sabemos que en total ser\'an $2^n$ distribuciones que a su vez, es el n\'umero de cadenas binarias de tama\~no $n$. Precisamente, en el algoritmo de \textit{fuerza bruta} lo que se hace es generar todas las cadenas binarias y luego, a partir de cada una, computar la distancia m\'axima en el conjunto resultante de proyectar los puntos $0's$ en $Y$ y los puntos $1's$ en $X$. Como se tienen $2^n$ cadenas y por cada una se recorre el conjunto de $n$ puntos para computar el di\'ametro m\'aximo, entonces, el algoritmo de \textit{fuerza bruta} es $O(n*2^n)$.\\
\textit{C\'odigo en: brute\_force.py}\\

Ahora, para empezar a mejorar la soluci\'on, vamos a demostrar el siguiente teorema sobre la respuesta \'optima al problema.\\

\textit{ \textbf{Teorema-Estructura \'optima:} Existe una distribuci\'on de puntos que garantiza el \'optimo, tal que, si se ordenan por las $x$, tiene la forma $S_yS_xS_y'$, siendo $S_y,S_x,S_y'$, subconjuntos contiguos de puntos.}\\
\textit{Demostraci\'on:} Sea $Op$ una distribuci\'on de puntos que garantiza el \'optimo. Si $Op$ no tiene puntos en $X$, hacemos  $S_x=\emptyset $ y dividimos los puntos $Y$ entre $S_y$ y $S_y'$, eventualmente vac\'ios y tenemos la estructura buscada.\\
Si $Op$ tiene un solo punto en las $X$, hacemos $S_x=\{x_i\}$, $S_y$ igual a los puntos menores que $x_i$ y $S_y$ igual a los mayores que $x_i$ y tenemos la estructura buscada.\\
Ahora, si hay al menos dos puntos que se colocan en el eje $X$, seleccionemos $Xl$ y $Xr$ los de mayor di\'ametro entre si. El resto de puntos proyectados en el eje $X$ est\'a contenido en el intervalo $[Xl,Xr]$ porque:
supongamos un punto proyectado en las $X$, $x_{out}$, que pertenece a $Op$ y no est\'a contenido en el intervalo anterior. Tenemos dos casos: $x_{out} < Xl$ o $x_{out} > Xr$. Si $x_{out}< Xl$, tenemos que $d(x_{out}, Xr) = d(x_{out}, Xl) + d(x_{out}, Xr) > d(Xl, Xr)$ 
lo cual contradice que el intervalo $[Xl,Xr]$ fuera el de mayor di\'ametro en $Op$. El otro caso es an\'alogo. \\
Tenemos que todos los puntos $X$ est\'an contenidos en el intervalo $Ix = [Xl,Xr]$. Si no hay elementos $Y$ en $Ix$ entonces decimos, $S_x=Ix$, $S_y$ las $y$ menores que $Xl$(por las x) y $S_y'$ las $y$ mayores que $Xr$. Ahora supongamos que en $Op$ hay alg\'un punto $Y$, digamos $y_i$ tal que $y_i$ est\'a en $Ix$. Como tenemos $[Xl,Xr]$ dentro del $Op$, sabemos que $d(Op)\geq d(Xl,Xr)$. Sabemos, por lo que se plantea en \textit{Relaci\'on de di\'ametros} cu\'ales di\'ametros definen el di\'ametro de $Op$, a saber: $d(max_x, min_x)=d(Xl,Xr)$, $d(max_y, min_y)$, $d(max_x, min_y)$, $d(min_x, min_y)$, $d(max_x, max_y)$ y $d(min_x, max_y)$.
Tenemos entonces $y_i$ contenido en el intervalo $[Xl,Xr]$, construyamos $Op'$ tal que el punto $y_i$ se proyecta en las $X$. Comprobemos que $Op'$ sigue siendo \'optimo. Proyectar $y_i$ en $X$, no afectar\'a el tama\~no de $d(Xl,Xr)$ porque est\'a contenido entre ambos extremos. Directamente tenemos que $d(Op)=d(Op')$ si $y_i\neq y\_max$ y $y_i \neq y\_min$ porque $y_i$ no estar\'ia involucrado en ninguno de los di\'ametros que definen en el m\'aximo. Vemos entonces, que si en otro caso $d(Op)\geq d(Op')$ porque eventualmente hemos disminuido uno de los extremos que definen los di\'ametros mayores, pero como $Op$ es \'optimo, no puede pasar que $d(Op)>d(Op')$ porque ser\'ia una contradicci\'on.\\
Entonces, proyectar $y_i$ en $X$ en cualquier caso mantiene $d(Op)=d(Op')$. Luego, tenemos que todo punto contenido en el intervalo $[Xl,Xr]$ puede ser proyectado en el eje $X$ sin afectar el \'optimo, y, por tanto, siguiendo una secuencia finita de la operaci\'on hecha con $y_i$ anteriormente, podemos convertir $Op$ en un $Op^*$, tambi\'en \'optimo, tal que en $Op^*$ el intervalo $[Xl,Xr]$ contiene solo puntos proyectados en el eje $X$ y, todos los puntos proyectados en el eje $X$, est\'an contenidos en $[Xl,Xr]$. Luego, hacemos $S_x=[Xl,Xr]$, $S_y$ las $y$ menores que $Xl$ y $S_y'$ las $y$ mayores que $Xr$. De esta forma tenemos la estructura planteada, con lo cual queda demostrado que existe un \'optimo con esta estructura, y que, de haber un \'optimo que no cumple dicha estructura, con una secuencia fininta de pasos se puede reducir a este.\\


\textit{Segunda soluci\'on:} Como sabemos que hay una distribuci\'on \'optima que tiene la estructura descrita en el \textit{Teorema-Estructura \'optima}, entonces la idea de esta soluci\'on es encontrar el intervalo $[Xl,Xr]$, ya que, una vez determinado los l\'imites de $S_x$, se pueden sabe que elementos pertencen a $Sy$ y $Sy'$.\\

\begin{verbatim}
    1-Ordenar los puntos por las x
    2-Computar los arreglos de prefijos y sufijos máximos y mínimos de las y.
    3-min_sd = min((max_x-min_x)**2,(max_y-min_y)**2)
    4-Por cada punto i de 0..n, elegirlo como Xl y:
        4.1 por cada punto de i a n elegirlo como XR:
            4.1.1 computar los diámetros de la distribución resultante.
            4.1.2 elegir el máximo y compararlo con min_sd y si es menor actualizar
    5-Devolver min_sd
\end{verbatim}
\textit{C\'odigo en: n2\_solution.py}\\

La soluci\'on hace lo siguiente: ordenar los puntos por su coordenada $x$, y marcar cada uno de ellos como inicio posible del intervalo $[Xl,Xr]$, o sea, como $Xl$, y explorar los restante puntos a su derecha como posibles $Xr$. Por cada uno de los intervalos que se forman, calculamos los di\'ametros que describen el conjunto que se forma, y nos quedamos con el mayor de ellos que ser\'a el di\'ametro de esa distribuci\'on. Todo el tiempo se mantiene una variable $min\_sd$, que guarda el menor di\'ametro de distribuci\'on que se ha visto. Como sabemos que existe un intervalo \'optimo, este tiene que ser el que da lugar a la distribuci\'on de menor di\'ametro de entre todas los que siguen esta estructura. Como por cada intervalo es necesario calcular los di\'ametros, es necesario conocer los puntos extremos en el eje $Y$ en esa distribuci\'on. Para poder obtener estos valores sin necesidad de iterar nuevamenente por el conjunto, luego de ordenar los puntos, se precomputan 4 arreglos con los posibles valores de las $y$ para cada intervalo: \textit{prefix\_min, prefix\_max, sufix\_min, sufix\_max} , que conitienen, dado un \'indice $i$, el valor m\'inimo, m\'aximo, m\'inimo, m\'aximo de las $y$ en los intervalos, $[0,i]$, $[0,i]$, $[[i,n]$ e $[i,n]$ respectivamente.\\
Precomputar cada uno de estos arreglos se hace en $O(n)$, iterando, para el caso de los prefijos, de $1$ a $n-1$ y haciendo $p[i] = min/max(a[i].y, p[i-1])$, luego de setear en $p[0]$ a $a[0].y$. Para el caso de los sufijos, se hace de $n-2$ hasta $0$ haciendo en cado paso $p[i]= min/max(a[i].y,p[i+1])$, luego de setear $p[n-1]=a[-1].y$. De esta forma, conocer $max/min$ $y$, se hace en $O(1)$ preguntando en los prefijos $max/min$ y sufijos $max/min$ por los elementos $Xl-1$ y $Xr+1$ respectivamente, y luego qued\'andose con el $max$ entre los $max$ y el $min$ entre  los $min$ obtenidos.\\

\textit{Complejidad temporal:} Tenemos el conjunto de puntos de tama\~no $n$. Ordenarlos por las $x$ es $O(nlogn)$, luego, precomputar los arreglos de prefijos y sufijos $max/min$ de las $y$ se hace en una iteraci\'on $1..n$ con operaciones $O(1)$, con lo cual tenemos que este prec\'omputo se hace en $O(n)$. La exploraci\'on del \'optimo se hace en dos ciclos $for$ anidados, uno que itera en $i$ de $0...(n-1)$, y otro que lo hace desde $j=i...(n-1)$, con lo cual, tenemos que se hacen:\\

$\sum_{i=0}^{n-1} n-i = n+(n-1)+....+1 = n(n+1)/2= (n^2+n)/2 = n^2/2+n/2$\\
iteraciones con operaciones constantes $O(1)$. Con lo cual determinamos el \'optimo en $O(n^2)$, y tenemos que el algoritmo al completo va a hacer $O(nlogn + n + n^2)=O(n^2)$.\\

\textit{Enfoque para la tercera soluci\'on:}  Como tenemos la soluci\'on acotada, o sea, tenemos el intervalo $D=[0,MAX]$, $MAX=min(D_2(min\_y,max\_y),D_2(min\_x, max\_x))$ donde est\'a contenido el \'optimo, se puede realizar una b\'usqueda binaria sobre mismo respondiendo al siguiente Predicado(D): El di\'ametro al cuadrado $D$ puede contener al conjunto?, O lo que es lo mimso: Hay alguna distribuci\'on tal que el $D_2$ del conjunto es menor igual que $D$?. Es necesario probar dos cosas, que el predicado es de la forma "no","no",....,"no","si","si",...,"si" y que, en ese caso el primer "si" es exactamente el $D_2$ \'optimo del conjunto.\\

Demostraci\'on del predicado: Sea $d_2$ un valor para el $D_2$, y sea $Opd$ el \'optimo del conjunto. Ning\'un $D_2 \leq Opd$ puede contener el conjunto, porque entonces ser\'ia  una contradicci\'on con que $Opd$ sea el \'optimo. A su vez, tenemos que para cualquier $d_2\geq Opd$ es posible encontrar una distribuci\'on del conjunto tal que el $d_2$ lo contiene, en especial podemos encontrar la de $Opd$ que verifica que el $D_2$ del conjunto, igual a $Opd$ es menor igual que $d_2$. Con lo cual, el predicado es de la forma "no","no",....,"no","si","si",...,"si" sobre los valores para el $D_2$ del conjunto y, en especial, hemos demostrado que $Opd$ es el primer "si", ya que es el menor valor que contiene al conjunto y, por tanto, el primero que lo hace.\\

Luego, la b\'usqueda binaria sobre el conjunto de posibles valores para el \'optimo es posible. Ahora, es necesario poder evaluar el Predicado de forma tal que se mejore la soluci\'on anterior.\\

Para evaluar el predicado necesitamos encontrar una distribuci\'on tal que que su $D_2$ sea $\leq D$, siendo $D$ el valor de la evaluaci\'on.\\

Las siguientes dos soluciones al problema siguen el mismo enfoque a la hora de evaluar el predicado, as\'i que, vamos a ver la parte te\'orica y luego el detalle de la implementaci\'on de cada una. Ambas parten de haber ordenado previamente el conjunto por las $X$ y de computar los arreglos de prefijos y sufijos $max/min$ de las $y$, tal como se hizo en la soluci\'on anterior.\\

\begin{verbatim}
    1-Ordenar los puntos por las x
    2-Computar los arreglos de prefijos y sufijos máximos y mínimos de las y.
    3-min_sd = min((max_x-min_x)**2,(max_y-min_y)**2)
    4-Busqueda binaria sobre el intervalo [0,min_sd]. Actualizar min_sd
    5-Devolver min_sd
\end{verbatim}


Tenemos por el \textit{Teorema-Estructura \'optima}, que existen las distribuciones que siguen la estructura $S_yS_xS_y'$, y que entre ellas hay una, la de menor $D_2$ que es \'optima para el problema. Entonces, y tambi\'en como resultado de la soluci\'on anterior, tenemos que si construimos todas las distribuciones que siguen esta estructura, eventualmente tendremos una que es \'optima y que por tanto su  $D_2=Opd$. Como no podemos iterar nuevamente de la misma forma que en la soluci\'on anterior, vamos a hacer lo siguiente: Sabemos que, en cualquier distribuci\'on de la forma descrita tendremos un intervalo $[Xl,Xr]$ y un $x\_min=min(|Xl|,|Xr|)$ y un  $x\_max=max(|Xl|,|Xr|)$. Entonces, vamos a marcar la posicion de $x\_max$ y para ella vamos a buscar la mejor $x\_min$ posible que valide a $D$ siendo $D$ el valor por el que se pregunta en el Predicado. Esto siempre va a ser posible para alguna $x\_max$, siempre y cuando, $D>=Opd$, ya que hay una $x\_max$ que genera el \'optimo. Si $D<Opd$, no va a ser posible, porque entonces habr\'ia una distribuci\'on que genera un $D_2$  mejor que el \'optimo, lo cual es una contradicci\'on.\\
Vamos a analizar la relaci\'on $x\_min$ $x\_max$ para una $x\_max$ arbitraria:\\
Tenemos que 1) $|x\_max|\geq|x\_min|$ por definici\'on y que para que valide a $D$, 2)$D_2(x\_max,x\_min)<=D$. \\ 
Tambi\'en el hecho de definir $x\_max$, hace que el menor $D_2$ que se pueda generar en el conjunto est\'e definido por los m\'aximo entre $D_2(x\_max,x\_min)$, $D_2(y\_max,y\_min)$ y $D_2(x\_max,y\_max)$, ya que la distancia de $x\_min$ a $y$ siempre ser\'a menor que la de $x\_max$, ya que tiene mayor valor absoluto y por tanto mayor distancia con el $(0,0)$ que, como vimos al principio, est\'a contenido en la distancia desde el eje $X$ hasta cualquier punto proyectado en las $y$. Con lo cual, tenemos que el mejor $x\_min$ para $x\_max$ va a ser el mayor que cumple 1 y 2, ya que no afecta los $D_2$ de las distribuciones.\\
A la hora de fijar las $x\_max$ va a tener dos posibles escenarios, o bien $x\_max=Xl$ y, por tanto, $<0$, o  $x\_max=Xr$ y por tanto $>=0$. Esto se debe a que, si estamos en el caso $Xl = max\_x$, si $Xl$ fuera $>0$, todos los puntos a su derecha cumplen que $|Xl|<|x_i|$ lo cual contradice que sea el $max\_x$. An\'alogamente ocurre con el caso $Xr$. Por tanto siempre ser\'a necesario manejar ambos escenarios a la hora de utilizar los valores de $x\_max$.\\


\textit{Soluci\'on con B\'usqueda Binaria}: Para validar el predicado para $D$, o sea, comprobar si $D>=D_2$ para  alg\'un $D_2$ posible en el conjunto lo que se hace es lo siguiente:\\
Se itera por el arreglo ordenados de puntos, y se va eligiendo cada uno como  $Xl$ mientras sea posible($points[l].x<0$), luego tenemos que buscar a su $x\_min$ tal y como se describi\'o antes. Para esto se hace una b\'usqueda binaria en todos los puntos a su derecha. Si no se encunetra una distribuci\'on que valida a $D$, entonces, an\'alogamente, se hace lo mismo pero iterando desde el extremo derecho $n-1$ a $0$, para el caso que en que $Xr=max\_x$\\
Analicemos ahora cual ser\'a el predicado de esta nueva b\'usqueda binaria. Sabemos que los $x\_min$ para un $x\_max$ cumplen las condiciones 1 y 2 descritas anteriormente, entonces, podemos ver que si se tiene un punto $xr$ que es v\'alido para  $Xl$, un punto $xr'$ a la izquierda de $xr$ tambi\'en lo ser\'a, ya que, si $|Xl|>=|xr|$ se cumple y $D_2(Xl, xr)<=D$ tambi\'en se tiene que se cumple, tenemos directo que $D_2(Xl,xr')$ se cumple(est\'a a la izquierda de $x_r$), y, para ver que $|Xl|>=|xr'|$, supongamos que eso no es cierto, entonces tendr\'iamos $|Xl| < |xr'|$ pero como $|Xl|<0$ y $xr$ est\'a a la derecha de $Xl$, entonces tendr\'ia que ser $xr'>0$ y por tanto, esto implicar\'ia que $xr>0$ porque $x_r'$ est\'a a la izquierda de $x_r$, y entonces se tiene $xr>xr'$ y por tanto $|Xl|>=|x_r|>|x_r'|>|Xl|$, lo cual es una contradicci\'on.\\
Se tiene que, si $x_r$ es v\'alido, $x_r'$ a la izquierda de $x_r$ tambi\'en  es v\'alido.\\ 
Entonces tengamos un $x_r$ a la derecha de $Xl$ tal que $Xl$ no es v\'alido para $x\_min$ respecto a $Xl$ y con $D$. Esto implica que $|Xl| < |xr|$ o $D_2(Xl, xr)>D$. Sea $xr'$ un punto a la derecha de $xr$, tenemos que si $D_2(Xl, xr)>D$, $D_2(Xl, xr')$ tabi\'en lo ser\'ia. As\'i que, supongamos que esa no es al condici\'on que invalida a $xr$, tiene que ser, por tanto, que $|Xl|<|xr|$, pero como tenemos que $Xl<0$, entonces $xr>0$ y por tanto $xr'$ a la derecha de $xr$ cumple que $xr'>xr>|Xl|$, con lo cual $|Xl|<|xr'|$ y se tendr\'ia que en cualquier caso, si $xr$ no es v\'alido, ning\'un $xr'$ a la derecha de $xr$ ser\'a v\'alido.\\
Luego, uniendo ambas situaciones tenemos que ante el predicado: Es $xr$ a la derecha de $Xl$ un $x\_min$ para $Xl$ que valida $D$?, la evaluaci\'on es de la forma "si","si",...,"si","no","no",..,"no". Entonces en el \'ultimo "si" tenemos el mejor $x\_min$ para $Xl$ que valida $D$.\\

El caso para cuando se elige $Xr= x\_max$, es sim\'etrico a este. \\

\begin{verbatim}
    #source va a ser el Xl/Xr elegido y target el xl/xr a #validar 
    def valid(source, target, D, points):
        if D_2(source, target) > D:
            return False
        if abs(source)<abs(target):
            return False
        return True
    
    def check(D, n, points):
        min_sd = INF
        for Xl in points[1....n]:
            if Xl > 0: break
            xr = Xl
            end = n
            while xr+ 1 < end:
                mid = (xr+end)//2
                if valid(Xl, mid, D, points):
                    xr = mid
                else:
                    end = mid
            Calcular los D_2.
            Actualizar min_sd
            if min_sd <= D:
                return True
        for Xr in points[1....n]:
            if Xr < 0: break
            xl = Xr
            end = -1
            while end+ 1 < xl:
                mid = (xl+end)//2
                if valid(Xr, mid, D, points):
                    xl = mid
                else:
                    end = mid
            Calcular los D_2.
            Actualizar min_sd
            if min_sd <= D:
                return True
        return False
\end{verbatim}
\textit{C\'odigo en: double\_bs\_solution.py}\\

\textit{Complejidad temporal}: Tenemos ya el $O(nlogn)$ de la ordenaci\'on, luego, hacemos una b\'usqueda binaria en el intervalo [0,MAX], y en cada paso hacemos $check()$. En $check()$, recorremos todas las $xl<0$ y se eligen como x\_max, y se hace una b\'usqueda por los restantes puntos. Si no se ha encontrado una distribuci\'on v\'alida a\'un, se hace el an\'alisis en el sentido inverso para las $xr>0$. En los casos que no haya validez para $D$ que son los peores escenarios para el algoritmo, se realizara la iteraci\'on sobre los n puntos y se har\'a una b\'usqueda binaria en los subconjuntos candidatos de m\'aximo $n-1$ puntos, los cual est\'a acotado por $n$. Entonces tenemos que buscar el \'optimo se har\'a en $O(logMAX*nlogn)$.\\

\textit{Soluci\'on con Dos punteros}: El mismo an\'alisis que se sigui\'o en la soluci\'on anterior se sigue en esta, lo que en este caso se utiliza en vez de una b\'usqueda binaria sobre los extremos, un segundo puntero para iterar por el $x\_min$. Vamos a ver el caso iterando desde $Xl$(el caso desde $Xr$ es sim\'etrico) .\\
\begin{verbatim}
xr = 1
for Xl in range(1,n+1):
    while xr<n && D2(points[Xl].x-points[xr+1].x)<= D &&|points[Xl].x|>=|points[xr+1].x|:
        xr+=1
    while |points[Xl].x|<|points[xr].x|:
        xr-=1
    Calcular los D_2.
    if max(D2s) <= D:
        return True
xl = n
for Xr in range(n,0,-1):
    while xl>1 && D2(points[xl-1].x-points[xr].x)<=D &&|points[xl-1].x|<=|points[Xr].x|:
        xl-=1
    while |points[xl].x|>|points[Xr].x|:
        xl+=1
    Calcular los D_2.
    if max(D2s) <= D:
        return True
\end{verbatim}
\textit{C\'odigo en: two\_pointers\_solution.py}\\

Sabemos que el mejor $x\_min$ para $Xl$ va a ser el \'ultimo que es v\'alido cumpliendo las dos condiones expresadas:\\
1) $|x\_max|\geq|x\_min|$ \\
2)$D_2(x\_max,x\_min)\leq D$. para que valide a $D$\\ 
Demostramos esto en el predicado de la b\'usqueda binaria.\\
Tenemos tambi\'en que, a medida que avance el puntero que selecciona $Xl$, los valores de $Xl_i$ se comportan de manera tal de que $|Xl_i|>|Xl_{i+1}|$ $\forall i$, porque se avanza hacia la derecha y solo en los negativos en un arreglo ordenado. \\
Entonces la cuesti\'on est\'a en ver como ocurren las transiciones en el puntero $xr$ que apunta a $x\_min$.\\
 Inicialmente se va a mover hasta el mejor $xr$ posible de $Xl_1$. Analicemos los casos en los que se detendr\'ia el puntero. 
 Primera condici\'on de parada, se alcanz\'o el final de los puntos, con lo cual no habr\'ia que analizar m\'as nada porque significa que el conjunto est\'a abarcado por las $X$ a partir de $Xl$.\\
 Si se llega a la segunda condici\'on de parada esta ocurrir\'a porque, si se mueve el puntero $xr$ a $xr_{+1}$, el $D_2$ de los extremos sobrepasar\'a $D$. Pero en este caso como el puntero comprueba que no va a poder cumplir la condici\'on para $D$, no avanza. Por tanto el que avanza es el puntero $Xl$, que al moverse desde $i$ a $i+1$ se mantendr\'a cumpliendo que la $D_2(Xl,xr)\leq D$ porque $|Xl_i|>|Xl_{i+1}|$ $\forall i$. Tenemos entonces que la condici\'on del di\'ametro al cuadrado nunca causa retroceso en el puntero de $xr$(derecha).\\
 Entonces, hay retroceso, cuando al avanzar  $Xl_i$ a $Xl_{i+1}$  ocurre que $|Xl_{i+1}|<|xr|$, con lo cual el puntero retrocede hasta el primer $xr'<xr$(por la derecha) que valida a $Xl_{i+1}$.  \\
 Pero una vez que el puntero retrocede ya nunca avanza m\'as, se mantiene o retrocede.\\
 
 Esto se cumple, porque, como hemos visto, $|Xl_i|>|Xl_{i+1}|$, y como el puntero retrocedi\'o a $xr'$ que cumpl\'ia\\ $|Xl_{i+1}|\geq|xr'|$, y cualquier otro $xr^{'}_{k}$ a la derecha de de $xr'$ va a cumplir que $|xr^{'}_{k}|>|xr'|$ y se tiene entonces que $|xr'|<|Xl_{i+1}|\leq |xr|$ y que cualquier $Xl'$ a la derecha de $Xl_{i+1}$ cumple que $|Xl_{i+1}|>|Xl'|$ con lo cual, una vez retrocede, no avanza de nuevo.\\
 
 La operaci\'on es an\'aloga cuando se hace desde la derecha(positivos), buscando el $x\_min$ hacia los negativos.\\
 
 \textit{Complejidad temporal}: Tenemos que el puntero no va avanzar m\'as de $n$ elementos ni tampoco va a retroceder m\'as de $n$ elementos. Con lo cual, tenemos que la cantidad de moviemientos del puntero para mantener los $xr$ es $O(n)$. Los $D_2$ se calculan en tiempo constante, con lo cual, el $check(D)$ ocurre en $O(n)$.\\
 
 Luego tenemos que la complejidad temporal del algoritmo va a ser $O(nlogn+nlogMAX)$, con lo cual queda en $O(nlog(max(n,MAX))$.
 
 
\section{Generador y tester}

En la carpeta \textit{.test/} se encuentran 2 subcarpetas \textit{in/, out/}, cada una contiene repectivamente, las entradas generadas y las salidas obtenidas por el algoritmo de fuerza bruta.\\
Tambi\'en se incluyen algunos casos tomados del tester del codeforce, estos se identifican por su nombre: \textit{$t<num>cf$}.\\

El generador de casos \textit{test/test\_generator.py}, genera juegos de datos de entrada de forma completamente aleatoria, siempre dentro de las restricciones del problema, o permite ingresar cotas manualmente, as\'i como especificar si se desea que todos los valores generados sean de un signo espec\'ifico(+ o -). La cantidad de puntos a generar est\'a restringida a $22$ por el $2^n$ de la fuerza bruta.\\

El generador coloca los archivos \textit{.in} y \textit{.out} en sus carpetas correspondientes.\\

Para validar las soluciones de los algoritmos, se utiliza el \textit{checker.py} que permite elegir la implementaci\'on a validar y da dos opciones para los casos de prueba: 1) Correr un caso espec\'ifico o 2) correr todos los casos.\\

La estrategia seguida para validar las soluciones fue la siguiente: Se leen los archivos de entrada y salida del test hechos por el generador y para cada uno se computa la soluci\'on del algoritmo seleccionado, se comprueba si el valor dado como \'optimo coincide.\\






\end{document}
